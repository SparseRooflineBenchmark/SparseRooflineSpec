\documentclass{article}
\usepackage{graphicx} % Required for inserting images
\usepackage{amsmath}
\usepackage{hyperref}

\usepackage[many]{tcolorbox}
\newcommand {\grace}[1]{{\color{violet}\sf{[Grace: #1]}}}
\newcommand {\willow}[1]{{\color{green}\sf{[Willow: #1]}}}
\newcommand {\olivia}[1]{{\color{red}\sf{[Olivia: #1]}}}
\newcommand {\dan}[1]{{\color{orange}\sf{[Dan: #1]}}}
\newcommand {\glb}[1]{{\color{blue}\sf{[Gilbert: #1]}}}
\newcommand {\jim}[1]{{\color{purple}\sf{[Jim: #1]}}}
\newcommand {\hrl}[1]{{\color{cyan}\sf{[Hengrui: #1]}}}
\title{The Sparse Roofline Benchmark Specification}
\author{Editors: Willow Ahrens, Olivia Hsu, Kyle Deeds, Grace Dinh (Please add your name!)}
\date{June 2023}

\begin{document}

\maketitle

\section{Introduction}

Sparse matrix and tensor problems have received increased attention in recent years. We have developed this benchmark suite to encourage meaningful comparison among the varieties of solutions.
Direct comparisons in sparse computing research are difficult due to the wide variety of sparse kernels, input-dependent performance, differences in intended use cases for each kernel, and a lack of realistic or theoretically interesting datasets.

Prior benchmark suites have encouraged competition to improve the performance of various subsets or combinations of sparse kernels. SparseBench was proposed by the Netlib to measure numerical improvements to sparse iterative solvers \cite{noauthor_sparsebench_nodate}. The GAP (Graph Algorithm Platform) benchmark suite describes entire graph applications composed of multiple sparse kernels \cite{beamer_gap_2017}. PASTA (A Parallel Sparse Tensor Algorithm Benchmark Suite) was proposed to measure sparse tensor kernels, but only concerns kernels with a single sparse tensor\cite{li_pasta_2019}. 
\olivia{I think we also need to address database collections of sparse data here, like SuiteSparse and FROSTT, and how people use them as pseudo benchmark data but it is incomplete. This will further place where our contribution is and motivate this project}

Our benchmark suite aims to measure a set of specific sparse kernels that completely covers the sparse performance engineering landscape. This requires several novel features.

\hrl{A bit unsure what kind of tensor generator we are looking into, maybe reference? Kernelized generator with pre-specified interactions for big tensor seems to be an open problem, like https://proceedings.mlr.press/v206/wesel23a.html.}
First, we specify datasets or generators for all of the inputs to a given kernel, including situations where multiple inputs are sparse and their patterns interact. As an example, this situation arises in sparse matrix times sparse matrix multiply, where the sparsity of the output is not known at the outset and may belong to different distributions depending on how the inputs correlate. 

Second, we specify explicit staging of sparse kernels to quantify the tradeoff between preprocessing time and runtime. Many optimizations to SpMV, for example, involve reformatting and reorganizing the matrix before multiplying it, under the assumption it will be used multiple times.

Third, we provide sparse tensor statistics for each problem (such as the sparsity of the output or the number of required operations), and generators for pathological cases. This helps quantify when implementations are reaching their theoretical limit, the roofline.

Our benchmark suite aims to provide researchers with the software and specifications to easily benchmark sparse problems on realistic datasets across multiple languages and architectures. We also provide an easily accessible database to tabulate the results: the best known approaches to sparse problems.

%Note: There is more potential text for the introduction at the end of the document

\section{Reference Implementation}

While the precise mathematical specification of the computation is specified in this document, we have also provided a reference implementation for each problem.
%We will produce a framework for running our benchmarks on various target hardwares, providing data for various backends. The framework will have places to plug in a candidate implementation, and the script will output the runtimes for appropriate systems.

\section{Tensor Formats} \label{sec:formats}

To enable cross-language compatibility, all tensors generators or dataset downloads will produce tensor files in Binsparse tensor formats \cite{noauthor_graphblasbinsparse-specification_nodate}. Possible tensor format names such as CSR or COO will refer to Binsparse format codes. Binsparse tensors are composed of several data vectors. On disk, these vectors may be stored in any array data file format such as an HDF5 container or an \texttt{.npy} file. During benchmarking, inputs and output data are expected to correspond to in-memory versions of these same arrays.

For example, the CSR structure involves three vectors, \texttt{pointers\_to\_1}, \texttt{indices\_1}, and \texttt{values}. The column indices of the stored values located in row i are located in the range [\texttt{pointers\_to\_1[i], pointers\_to\_1[i+1]}) in the \texttt{indices\_1} array. The scalar values for each of those stored values is stored in the corresponding index in the values array. On disk, these arrays might be serialized to a text representation, but in memory in C, the data would be referenced by two pointers to integers and one pointer to floating-point numbers.

The precise filenames and paths will be specified along with the benchmark kernel that uses them. For example, one might use the RMAT matrix generator included with the suite to generate inputs, stored in \texttt{spmv/RMAT/A.bsp.h5} and \texttt{spmv/RMAT/x.bsp.h5}
\hrl{In h5 format, especially in those sparse data filled with zeros, practitioner will use a low-rank approximation (e.g., 'On the Compression of Low Rank Matrices' by Cheng et.al) to store it at the cost of minimal loss of accuracy.}
Subsequent sections of this document may specify that inputs live in ``dense'' formats or ``sparse'' formats. We say that the dense binsparse formats are those that store all possible entries, and sparse formats are those that allow one to store only a subset of the entries.

\willow{There are more possibilities here, we orthogonalize concerns to leave discussion to another committee...}

\section{Timing}

The time to transfer input/output data to and from the input file should not be measured.  Timing should begin with inputs resident in memory with the specified binary format (see Section~\ref{sec:formats}). The timing should end after the output is resident in memory with the specified binary format. This benchmark may apply to a diversity of architectures with different memory hierarchies, from distributed memories to special purpose hardware accelerators. The only requirement is that the outputs are resident in the same memory as the inputs, and that only one copy of the input is used. \grace{Some clarification about how to handle this in the case of accelerators with distinct scratchpads and accumulators should be done. Do scratchpads and accumulators count as 'the same' memory?}

Some accelerators have high bandwidth costs to and from the chip (making timings for problems resident in accelerator memory less realistic for those who wish to use the data off-chip). However, this may be accounted for naturally as the problem size grows to exceed the on-chip memory.

\subsection{Staging}

Each benchmark is separated into stages based on which inputs the application is allowed to read at each stage. For example, to enable a clear distinction between matrix preprocessing and runtime in SpMV kernels, the first stage allows access to just the matrix. Any vector-agnostic preprocessing can be timed within the first stage. The second stage allows access to the vector, and thus computation of the output. We recognize that many divisions between stages are possible; we have selected separations into stages that we believe are meaningful for applications.

\subsection{Results Aggregation}

\subsubsection{Deterministic data}

    In situations where the input sparsity patterns are fixed, applications may run as many executions as they wish, reporting the minimum time. Applications should report whether the cache was cleared at the start of each execution.

\subsubsection{Randomized data}

    In situations where the input sparsity patterns are drawn from a distribution, applications must choose a random seed for the generator, and report the mean and variance of the execution times, as well as confidence intervals for each statistic. Implementations are free to choose how many executions are used, but a confidence interval of at least 95\% is required for the mean time.

    The software implementation provides a convenient command-line utility to compute these statistics, as well as an in-software utility, so implementers need only record the runtime on each random input.

    For each random trial input, implementers should follow the guidelines for deterministic data, taking the minimum of as many execution times as they wish and reporting whether the cache was cleared before each.

\section{Correctness and Validation}

    All kernels in the benchmark suite are composed of one or more statements of the form
    \begin{equation}\label{eq:einsum}
        A_{i...} = \sum B_{j...} \cdot C_{k...} \cdot ... ,
    \end{equation}
    where $i, j, k$ may contain overlapping indicies and the operators $+$ or $\cdot$ may be substituted with other operators.

    \subsection{Allowed Optimizations}
        Reference implementations are provided for all kernels in the benchmark suite. Historically, we recongize that many optimizations have been proposed which are typically excluded from benchmarks, such as Strassen's algorithm. However, a goal of this benchmark is to encourage exploration of the unique properties of sparse kernels that arise from the constraint that one must do "all the same operations" as the reference implementation. We therefore add a category for limited optimizations, and considerations for those who wish to perform further algorithmic optimizations.

    \subsubsection{The Semiring Category}
        Both theoretical lower bounds and popular implementations of sparse kernels, such as GraphBLAS, have noticed that a wide variety of sparse optimizations can be described by the mathematical construct of semirings. Thus our "limited" optimizations may utilize any property derived from operators forming semirings. This is the default category we expect most optimizations to fall under, and includes the associative and commutative property. This does not include the distributive property.  In this categorization, we treat floating point numbers as real numbers for the purpose of applying mathematical properties.
    \subsubsection{Beyond Semirings}
        For implementations that wish to break the above rule, any mathematically equivalent optimization is allowed. This category allows optimizations such as Strassen's algorithm. 

    \subsection{Floating-Point Accuracy}
        Although the previous section treats floating point numbers as real numbers, in practice they behave differently. We do not require any particular exception handling behavior.
        
         \subsubsection{Error Bounds}
            Implementations involving floating point numbers must satisfy a relative error bound on \eqref{eq:einsum} of the following form, where $\epsilon$ is a declared parameter and $n$ is the number of elements in the sum:

            \[
                \forall_{i...} \left|A\_ref_{i...} - A\_res_{i...}\right| \leq n \epsilon \sum |B_{j...} \cdot C_{k...} \cdot ...|
            \]

            If only \texttt{Semigroup} level optimizations are used, then this equation will hold for the default value of machine $\epsilon$.

            This equation is inspired by the standard error bound for recursive summation, taken from \cite{higham_accuracy_1993}, and our generalized form can be adapted to several strategies for summation.

            \subsection{Floating Point Accuracy V2 \jim{This draft is based on the current draft of the Sparse BLAS Standard}}

            The emergence of various arithmetic formats, including mixed precision,
is making it increasingly important for numerical software to carefully
document what accuracy and related properties users can expect,
and for standards committees to evaluate these
claims, through extensive test code. Based on existing formats and
algorithms, we propose 2 error bounds, each depending on 2 parameters.
The two 2 error bounds correspond to using either conventional floating point,
or some variant of block floating point, to accumulate the (sparse) dot products
underlying those SparseBLAS operations that are expressible using 
dot products (we assume no Strassen-like algorithms are used, since these
make the intermediate results much less sparse). The two parameters 
correspond to the values of machine epsilon used to accumulate the dot product, 
and to store the final result, which may be lower in the case of mixed precision. 
The error bounds can of course depend both on the operation, and the input and 
output data types.

We express the overall error bound as follows. $E$ represents the exact result
(of a dot product), $A_1$ the first approximation, and $A_2$ the final approximation,
which may equal $A_1$, or require rounding $A_1$ to lower precision.
Suppose we can bound $|E - A_1| \leq B_1$ (an absolute error bound),
and $|A_2 - A_1| \leq B_2 \dot |A_1|$ (a relative error bound, so $B_2 \ll 1$). 
Then the overall bound is
\begin{eqnarray*}
|E - A_2| & \leq & |E - A_1| + |A_1 - A_2| 
            \leq B_1 + B_2 \cdot |A_1|
            \leq B_1 + B_2 \cdot (|E| + B_1) \\
        & = & B_2 \cdot |E| + B_1 \cdot (1+B_2)
            \approx  B_2 \cdot |E| + B_1
\end{eqnarray*}
since $B_2 \ll 1$.
In most cases, either $B_2 = 0$ (without mixed precision) or $B_2$ is the
machine epsilon of the output format. Letting $\sum_{i=1}^n x_i \cdot y_i$
be the exact dot product, $B_1$ will either be
\[
   B_{11} = f(n) \cdot \epsilon \cdot \sum_{i=1}^n | x_i |  \cdot  | y_i |
\]
if floating point arithmetic (with machine precision $\epsilon$) is used 
to compute the dot product, or
\[
   B_{12} = g(n) \cdot \epsilon \cdot \max_i | x_i |  \cdot  \max_i | y_i |
\]
if some kind of block floating point is used, including the technique
exploiting integer arithmetic in arxiv:2306.11975v3.
The functions $f(n)$ and $g(n)$ are often $n$ in the
worst case, but can also be supplied by the sparse BLAS
developer.

It is not difficult  to see how to generate vectors $x$ and $y$ where $B_{11} \ll  B_{12}$,
to distinguish these two cases. More thought is required to design
vectors $x$ and $y$ where different variants of block floating point will have
errors closer to $B_{12}$ than $B_{11}$.

Other issues to possibly consider are (1) when $A$ in $A \cdot x$ is stored
in CSC format using block floating point, so that the inner loop is doing a
sparse axpy, and (2) when the common exponent in block floating point is
shared over more than 1 row or 1 column of $A$.

\subsubsection{Exception Handling}

Other numerical properties to be tested include the range of numbers
that can be handled without over/underflow, and whether NaNs and
Infs propagate as expected (arxiv:2207.09281).

One difference between the dense BLAS and sparse BLAS
is exception propagation with implicit zeros.
Implicit zeros in a sparse matrix A should probably not create NaNs when (not) performing
the multiplication A(i,j)*x(j) = 0 * Inf or 0 * NaN. This seems consistent with user expectations,
and analogous to how we treat alpha = 0 or beta = 0. But if an optimization like register blocking
turns an implicit zero into an explicit zero, then the NaN might propagate. I think this is ok for
the default behavior, possibly propagating exceptions "too much" but with the best performance.
We could discuss alternatives, like scanning x (in A*x) for Infs and NaNs, and doing something
more carefully in the very rare event that one appears, and providing this as a user selectable option.
Similar considerations apply to masked operations, if we choose to include those. We could also consider a third
user option, guaranteed bitwise-reproducility, including
exception propagation, which could be much slower, but
useful for debugging.

We should also consider how to handle zeros in x in TRSV: solving T*x = b. In the old reference BLAS, TRSV sometimes did and
sometimes did not check for zeros in x to avoid the work of multiplying the i-th column of T by x(i)=0.
These x(i)=0 could result either from trailing/leading zeros in b when T is upper/lower triangular,
or from exact numerical cancellation (much rarer). In the dense case we proposed disallowing
checking for zeros in x, and providing a simple routine to count the number of leading/trailing
zeros in b in case the user wanted to take advantage of this optimization. We should think about
these options when we discuss sparse TRSV, since x(i)=0 could occur more frequently in the sparse case. 

\section{Datasets}

What makes a choice of dataset a good choice?
\begin{itemize}
    \item There is a citation we can make
    \item There is a clear story/good argument connecting this data and its use (on a given kernel) to an application that people care about.
    \item The dataset is taken from another already existing benchmark
    \item The dataset exercises a particular extremal complexity case (e.g. expander graphs, or things coming from the lower bounds group)
    \item That the dataset should be representative of a real application of a  kernel (i.e. make sense together)
\end{itemize}

\subsection{Tensor Datasets}

	Tensors may be generated by a generator, or downloaded from a dataset
	online. The tensor datasets are listed here, together with their keys. For
	example, the key \texttt{SuiteSparse/Boeing/ct20stif.bs} refers to the
	\texttt{Boeing/ct20stif} matrix in the \texttt{SuiteSparse} collection. The
	\texttt{download.jl} script included with the suite will download and reformat
	datasets to binsparse or TTX format.

	\begin{itemize}
		\item[\texttt{SuiteSparse}] The University of Florida SuiteSparse Matrix Collection \cite{davis_university_2011}
		\item[\texttt{FROSTT}] The Formiddable Repository of Open Sparse Tensors and Tools \cite{smith_frostt_2017}
		\item[\texttt{ImageNet}] \url{https://www.kaggle.com/competitions/imagenet-object-localization-challenge/}
	\end{itemize}
        \hrl{I suggest fMRI datasets and HMP datasets, https://adni.loni.usc.edu/, which are kind of popular and known to be lacking this kind of benchmark}
\subsection{Tensor Generators}

    Tensor generators are listed here, together with a description of their
    parameters. All tensor generators that use random numbers additionally take a
	seed argument $s$. 

\subsubsection{RMAT}
    The RMAT matrix generator \cite{chakrabarti_r-mat_2004} creates graphs with
    power-law degree distributions, which presents an interesting challenge to
    load balancing for parallelization. The generator works by taking the
    Kronecker product of a seed matrix with itself several times to produce a
    matrix of probabilities of including nonzeros. An artifact of RMAT matrix
    generation is that heavy rows are often grouped together (a "staircase"
    pattern), so we choose to apply a random row/column permutation to the rows
    and columns. RMAT is used in the Graph500 and GAP benchmarks
    \cite{noauthor_graph_nodate, beamer_gap_2017}. Additionally, the tensor
    generalization (which takes Kronecker products of a tensor seed with itself)
    is used in the PASTA Benchmark Suite \cite{li_pasta_2019}.

    The RMAT matrix generator classically uses a generating matrix

	\[
		G = \left[\begin{array}{cc}
			A & B \\
			C & D \\
		\end{array}\right],
	\]
	though we will specify the entire tensor $G$, rather than the parameters
	$A$, $B$, $C$, $D$. Additionally, we will specify $r$, the number of seed
	matrices which should be joined with a Kronecker product.

\section{Kernels}

\subsection{SpMV}
Sparse Matrix Vector multiplication (SpMV) \hrl{abbr.}
\subsubsection{Input Arguments}

\begin{itemize}
	\item[$m$:] a positive integer.
	\item[$n$:] a positive integer.
	\item[$A$:] an $m \times n$ sparse matrix.
	\item[$x$:] an $n$-length dense vector.
\end{itemize}

\subsubsection{Output Arguments}

\begin{itemize}
	\item[$y$:] an $m$-length dense vector.
\end{itemize}

\subsubsection{Computation}

$$y_i = \sum_{j}A_{ij}x_j$$

\subsubsection{Sparse Format Variants}

\glb{This text should be reduced.  It's just here to suggest}
There should be a list here of options that have already been defined in the format section for $A$ and $x$ (and any other inputs).  In the case of SpMV, this just means specifying format for $A$, because $x$ and $y$ are dense.

\willow{This is the section that would specify whichever formats we think are interesting for SpMV.}

\subsubsection{Benchmark Stages (Provide timings for each stage)}

SpMV is frequently used repeatedly on the same matrix. Therefore, it is the
subject of many optimizations which are independent of the vector $x$. Thus, 
we separate the benchmark into two stages:

\begin{itemize}
\item \textbf{(Stage 1)}
	The kernel may read $m$, $n$, and $A$.
\item \textbf{(Stage 2)}:
	The kernel may read $x$.
\end{itemize}

\subsubsection{Datasets}

\begin{itemize}
	\item[\texttt{spmv/kronecker10}:] 100 random kronecker networks at 10 percent sparsity, using the
	same parameters as GAP and Graph500 \cite{noauthor_graph_nodate}.
	\begin{itemize}
		\item[\texttt{spmv/kronecker10/A}:] $\texttt{RMAT}\left(r=27, G = \left[\begin{array}{cc}
			0.57 & 0.19 \\
			0.19 & 0.05 \\
		\end{array}\right]\right)$
		\item[\texttt{spmv/kronecker10/x}:] $\texttt{rand}\left(m = 2^{27}\right)$
	\end{itemize}
	\item[\texttt{spmv/OSKI}:] the sparse matrices used in evaluation of
	``Performance Optimizations and Bounds for Sparse Matrix-Vector Multiply'' \cite{vuduc_performance_2002}.
\end{itemize}

\subsubsection{Variants to Measure}
\hrl{We perhaps want to select some problems with ground truth so that we can have the actual error.}
Here are the combinations of settings that should actually be measured.


Everything after this line is a draft:

\noindent\rule{\textwidth}{1pt}

\subsubsection{Variations}
    Also include SpMV transpose.

    Mention $A=A^T$, possible additional optimizations,
    including``Fast Bilinear Algorithms for Symmetric Tensor 
    Contractions'', E. Solomonik, J. Demmel, Computational Methods
    in Applied Mathematics, Feb 2020.
    \newline
    https://doi.org/10.1515/cmam-2019-0075



\subsection{Iterative (Jacobi? PCG?) Solve}
	Input (Stage 1):
		Matrix given (e.g. “Boeing/ct20stif” or power-law graph laplacian)
	Input (Stage 2):
		Vector (e.g. uniform random dense vector)
	Compute:
		$y_j = A_ij * x_j$
	Timings:
Report time start stage 1 to start stage 2
Report time start stage 2 to end
Report aggregate time
Some combination of:
1. Allow reorganization of algorithm up to equivalence under arbitrary semirings (treat unsupported functions as uninterpreted) (e.g. thou shalt not do Strassen)
2. Allow reorganization of algorithm up to equivalence under infinite precision
2.5 (2 and 3)
3. Anything you want up to declared empirical accuracy
If algorithm is iterative, run steps until tolerance
If algorithm is non-iterative, report achieved empirical accuracy compared to double.

For SpMV, we compare to the naive implementation and take the L2-norm of the residual. 




\subsection{SpMSpV}
``Sparse-Sparse Matrix-Vector Multiply'' i.e. the matrices and vectors are all sparse.
$$y_i += A_{i,j} * x_j$$

\subsection{SpMM (SpGEMM?)}
	Input (Stage 1):
		Matrix
	Input (Stage 2):
		Vector
	Compute:
		$y_ij = A_ik * x_kj$
	Timings:
Report time start stage 1 to start stage 2
Report time start stage 2 to end
Report aggregate time

SpGEMM from multi-source BFS:
	Computation: Compute BFS parents starting from a particular set of vertices.
	Distribution: If the graph is fixed, you can precompute the BFS parents (not interesting). Thus, this problem is only well-posed on graph generators. I’m not sure if the set of vertices matters, could pick randomly.

SpGEMM from setup phase of algebraic multigrid:
	Computation: Compute $R_lA_lP_L$ as described in \url{https://doi.org/10.1145/3571157}.
	Distribution: matrices described in \url{https://doi.org/10.1145/3571157}
SpGEMM with various levels of size prediction difficulty and output reuse:
	Computation: Compute C = A*B
	Distribution: Generate A and B (using RMAT, perhaps?) so that expected number of nonzeros in A and B remain fixed but expected number of nonzeros in C can be very high, very low, or highly variable

\subsection{ATA}

\subsection{MTTKRP}
	Computation: A single MTTKRP 
	Distribution: The dense matrices are random, but the tensor can be fixed or random.


\subsection{Candecomp (deprecated)}
	Computation: Some fixed number of MTTKRPs of the same tensor in alternating transposition orders on the same factor matrices
	Distribution: The tensor can be fixed or random here as well.
Stencil (Same as SpMV, but matrix is a stencil)
	Computation: SpMV or Congugate gradient
	Distribution: Matrix is a toeplitz matrix for a particular stencil, vector is random

\subsection{FFT (deprecated)}
	Computation: SpMV
	Distribution: Matrix is an FFT matrix, vector is random

\subsection{Dynamic PageRank (deprecated)}
	Computation: Compute Pagerank of a matrix
	Stage 1.
		The first (n - d) rows and columns of a matrix are given
	Stage 2.
		The final d rows and columns are given
	Computation:
		Do 20 iterations of pagerank
	Notes: This incentivizes the implementation to use the first portion of the matrix to accelerate the computation in stage 2.
(Gilbert: This is just a power iteration? Willow: It is. I was trying to emphasize the dynamic graph kernel problem, where updates to a graph are received in real time and we are trying to maintain some statistics on the graph as the updates come in. Gilbert: I'm not sure page-rank or similar are incrementalized in practice?  Maybe.  Maybe worth considering social network clustering algorithms here instead?)

\subsection{Quantum Circuits (Deprecated)}
	Computation: Compute the distribution of the quantum circuits outputs
	Distribution: 1 Tensor for each circuit element (10s-100s total)
Transformer Model:
	Computation: Compute the output of a single encoder block in a transformer model with k attention heads
	Distribution: Randomly initialized weights with some sparsification algorithm applied
 
 \begin{align*}
Q[S, E, K] &= \sum_C(I[S, C] * WQ[C, E, K])\\
K[S, E, K] &= \sum_C(I[S, C] * WK[C, E, K])\\
V[S, E, K] &= \sum_C(I[S, C] * WV[C, E, K])\\
AP[S, S’, K] &= \sum_E(Q[S, E, K] * K[S’, E, K]) / 8\\
Z[S, E, K] &= \sum_S’(Sigmoid(AP[S,S’,K]) * V[S’, E, K])\\
CZ[S, E] &= \sum_k(Z[S, E, K] * WZ[K,E])\\
FF[S,FF] &= Relu(CZ[S,E] * WFF1[E,FF])\\
O[S,E] &= Relu(FF[S,FF] * WFF2[FF,E])\\
\end{align*}

\subsection{CountSat (Deprecated?)}
	Computation: Compute the number of satisfying solutions for a sat problem. This is a tensor contraction along each variable of the product of all clause tensors according to matching variables.
	Distribution: Randomly generated sat clauses with variables of length 3-10 (Existing Sat datasets? Are random sat problems unsatisfiable w.h.p.?)

\subsection{Graph Neural Networks (TODO)}
	Computation: Compute one iteration of message passing
	Distribution: Random graph models, random sparsified weights

% WX, BX is a simple affine transform, but it could be NN layers instead
$NeighborMessages[u, v, k’] = A[u, v]*(Sum_k(X[v, k]*WX[k, k’]) +BX[k’])  $

$AggNeighborMessages[u, k’] = Sum_v(NeighborMessage[u, v, k’])$

% This could also be a neural network
$NewX[u, k] = softmax_k(WX’[u,k](X[u, k] +AggNeighborMessages[u,k]) + BX’[k]) $

\subsection{SubGraph Counting (Deprecated?)}
	Computation: Compute the number of subgraphs of a particular kind (triangle, flower, petal, etc.)
	Distribution: Random graph models, directed vs undirected subvariants

\section{Future Benchmark Verticals}

\subsection{Quantum Chemistry Stuff}
Oh cool, quantum chemistry stuff!

\subsection{Morgue from Earlier Notes}

Produce a List of all Computation Kernels (Acronyms & Definitions)
SpMV? SpMM? SpDM3? SpMSpV? SPGEMM? SpLU? SpTRSV/M?

SpMV:
	“Sparse Matrix-Vector Multiply”
Dense x,y, Sparse A
y[i] += A[i, j] * x[j]
SpMSpV:
	“Sparse-Sparse Matrix-Vector Multiply”
Sparse A, x, y
y[i] += A[i, j] * x[j]
SpGEMM:
	“Sparse-Sparse Matrix Multiply”
Sparse C, A, B, often all matrices are square
C[i, j] += A[i, k] * B[k, j]
Masked SpGEMM:
	Sparse (sometimes Boolean) S, C, Sparse A, B
C[i, j] += S[i, j] * A[i, k] * B[k, j]
SDDMM:
	“Sampled Dense Dense Matrix Multiply”
	Sparse (sometimes Boolean) S, C, Dense A, B
C[i, j] += S[i, j] * A[i, k] * B[k, j]

SpMM:
	“Sparse Matrix Multiply”
Dense C, B, Sparse A, often J small constant
C[i, j] += A[i, k] * B[k, j]

MTTKRP:
	“Matricized Tensor Times Khatri-Rao Product”
Sparse B, Dense A, C, D
A[i, j] += B[i, k, l] * C[k, j] * D[l, j]
SpTTM:
	“Sparse Tensor Times Matrix”
Sparse A, B, Dense C, usually K >> L or L >> K
A[i, j, k] += B[i, j, l] * C[k, l]
SpMTTKRP:
Sparse A, B, C, D
A[i, j] += B[i, k, l] * C[k, j] * D[l, j]
SpLU:
	“Sparse LU Factorization”
Sparse A, produce permutations P, Q, sparse lower triangular L, and sparse upper triangular U
such that PAQ=LU
SpTRSV:
	“Sparse Triangular Solve”
	Sparse triangular matrix A, dense vector y, produce dense vector x such that Ax = y
SpTRSM:
	Sparse triangular matrix A, dense matrix Y, produce dense matrix X such that AX = Y

Willow: I’ve taken a stab at listing some kernel definitions. One might group them broadly by:
	1 argument sparse: SpMV, SpMM, SDDMM,MTTKRP, SpTTM
	2+ argument sparse: SpGEMM, SpMSpV, SpMTTKRP
	Loop carry dependency: SpLU, SpTRSV, SpTRSM

%comment: Here is a copy of the call for comments on this benchmark suite, perhaps we can work it into a motivation section above
%Goals the the benchmark suite:
%To discover the best possible approach
%A well-specified benchmark suite that specifies all relevant aspects of problem setup and measurement to facilitate meaningful comparison
%To provide a complete set of problems and data that exercise all important phenomena of sparse kernels and all realistic applications of sparse kernels.
%Ease of use and encouraging participation across a wide variety of participants, from tensor compiler writers targeting many kernels to hardware developers of a single kernel
	
%Why bother doing this?
%We are providing specifications for repeatable benchmarking, accuracy under reduced precision or semirings, and accounting for preprocessing time. 

%Some other people have created tensor benchmarks: PASTA, GAP, Sparsebench
%Why aren’t these sufficient?
%Well, the people on this project haven’t found them sufficient for the papers they published recently (or did they not even try?)
%PASTA doesn’t have a big enough variety of tensors
%All existing benchmarks only provide a single (sparse) matrix or tensor, but most real problems require more than one input.
%Most sparse benchmarks are focused on providing example data to the detriment of specifying computations and measurements
%Sparse problems are highly dependent on format/data-structure choices and/or preprocessing.  (maybe talk about measures instead?)
%Existing benchmarks 
%Current benchmarks do not collect and archive community progress on key problems.
%Are the problems in current benchmarks bullshit problems that no-one really cares about?  (Are ours?)
%The existing datasets are very good for certain sparse applications, but the variety of applications and approaches today are not well represented.
%List of specific, surprising, deficiencies
%There is not sufficient focus on exploring theoretical limits???
%We have an intellectual interest in the best real-world solutions with architectures that exist today.
%The dense world has a roofline model, it implies a certain workflow for crafting dense kernels and focuses development. The sparse world has no model, we don’t know when to stop optimizing. 
%What are the key problems that distill the essence of what “sparse linear/tensor algebra” is about?

%This will benefit the community because it will make it easier to measure progress by having common benchmarks for projects and papers.

%Our benchmark will include a few key features to support our goals: 
%We’re interested in a broad set of benchmarks; our initial kernels include SpMV, SpMM, SpGEMM, SpMTTKRP, Iterative Solvers, Subgraph Counting, and more!
%We will distribute simple, scalable reference implementations for our broad catalog of kernels. We will also include convenient code to generate inputs and compare against the reference.
%We will curate the results online and provide performance metrics that can be used to assess fractions of peak, such as best-known modeled data movement, matrix partitions, and more!

%Here is the bibliography:
\bibliographystyle{plainurl}
\bibliography{main}

\end{document}
